\documentclass[a4paper, 12pt]{book}

%%%%%%%%%%%%%%%%%%%%%%%%%%%%%%%%%%%%%%%%%%%%%%%%%%%%%%%%%%%%%%%%%%%%%%%%%%%%%%%%
% Packages

% Languages
\usepackage[ngerman,british]{babel}
\usepackage[autostyle,english=british]{csquotes}
\usepackage[super]{nth}

% Fonts
\usepackage{fontspec}
\usepackage{libertine}
\usepackage{anyfontsize}
\usepackage{musicography}

% CC licence
\usepackage[
    type={CC},
    modifier={by-sa},
    version={4.0}
]{doclicense}

% Include PDF and EPS files
\usepackage[final]{pdfpages}
\usepackage{graphicx}

% Links
\usepackage[colorlinks=true, allcolors=blue]{hyperref}
\usepackage{relsize}
\renewcommand*{\UrlFont}{\ttfamily\smaller\relax}
\usepackage{bookmark}

% Bibliography
\usepackage[backend=biber,notes,language=british]{biblatex-chicago}
\addbibresource{bibliography.bib}

%%%%%%%%%%%%%%%%%%%%%%%%%%%%%%%%%%%%%%%%%%%%%%%%%%%%%%%%%%%%%%%%%%%%%%%%%%%%%%%%
% Parsing of annotations extracted by scholarLY

\usepackage{pgfkeys}
\usepackage{ifthen}
\usepackage{etoolbox}
\usepackage{listofitems}
\setsepchar{,}

\pgfkeys{
  /ann/.is family,
  /ann,
  message/.store in=\annMessage,
  measure-no/.store in=\annMeasureNo,
  measure-pos/.store in=\annMeasurePos,
  beat-string/.store in=\annBeatString,
  beat-fraction/.store in=\annBeatFraction,
  beat-part/.store in=\annBeatPart,
  our-beat/.store in=\annOurBeat,
  rhythmic-location/.store in=\annRhythmicLocation,
  meter/.store in=\annMeter,
  source/.code={\pgfkeyssetvalue{source}{#1}\pgfkeysgetvalue{source}{\annSource}}
}

%%%%%%%%%%%%%%%%%%%%%%%%%%%%%%%%%%%%%%%%%%%%%%%%%%%%%%%%%%%%%%%%%%%%%%%%%%%%%%%%
% Formatting commands

% The name of a source
\newcommand{\source}[2]{\textbf{#1\textsubscript{#2}}}

% The \criticalRemark command used in annotations extracted by scholarLY
\newcommand{\criticalRemark}[1][]{%
  \pgfkeys{/ann,#1}% Parse the arguments
  \paragraph{Measure \annMeasureNo, beat \annOurBeat}
  \pgfkeysifdefined{source}{
    \ifthenelse{\equal{\annSource}{}}{}{
      \readlist\annSourceList{\annSource}
      Preferred \ifthenelse{\equal{\annSourceListlen}{1}}{source}{sources}: \annSource.
    }
  }{}
  \par \annMessage
}

\begin{document}

\frontmatter

%%%%%%%%%%%%%%%%%%%%%%%%%%%%%%%%%%%%%%%%%%%%%%%%%%%%%%%%%%%%%%%%%%%%%%%%%%%%%%%%
% Title page

\begin{titlepage}
\begin{center}
  \fontspec{Quattrocento-Bold.otf}{\fontsize{40}{50}\selectfont Felix Mendelssohn} \\
  \fontspec{Quattrocento.otf}\vspace{2cm}
  {\fontsize{32}{42}\selectfont Prelude} \\
  \vspace{2 cm}
  {\fontsize{24}{34}\selectfont MWV U 123} \\
  \vspace{0.5 cm}
  {\fontsize{24}{34}\selectfont Op. 104a, No. 2} \\
  \vspace{4 cm}
  {\Large \ifdef{\critical}{Critical Edition}{Performance Edition \ifdef{\variant}{(Variant)}{}}} \\
  \vspace{0.5 cm}
  {\Large Edited by Benjamin Geer} \\
  \vspace{4.5 cm}
  {\footnotesize Version 0.2 (draft), 10 February 2021} \\
  \vspace{0.1 cm}
  {\footnotesize Copyright \copyright\ 2020 Data and Service Center
    for the Humanities \\ \doclicenseText} \\
  \begin{minipage}{\textwidth}
  \centering
  $\vcenter{\hbox{\includegraphics[height=1.5cm]{DaSCH_Logo_CMYK}}}$
  \hspace*{.5 cm}
  $\vcenter{\hbox{\doclicenseImage[imagewidth=2cm]}}$
  \end{minipage}
  \end{center}
\end{titlepage}

%%%%%%%%%%%%%%%%%%%%%%%%%%%%%%%%%%%%%%%%%%%%%%%%%%%%%%%%%%%%%%%%%%%%%%%%%%%%%%%%
% Main document content

\chapter*{Preface}

This draft edition is part of an experimental project,
\url{https://tondauer.art}, aimed at developing technology for digital
editions of music.
\ifdef{\critical}
  {% Text for the critical edition
    An interactive online edition will eventually be based on this
    one. Performance editions based on this edition can be found on
    the project web site.
  }{% Text for the performance edition
    
    Footnotes in the music text point out editorial choices that are
    worth your attention, especially where you may prefer a different
    option. The critical edition, which can be found on the project
    web site, explains these choices.

    % TODO: say something about the fingerings.
  }

  \ifdef{\variant}
    {% Text for the performance edition with a variant reading
      This variant edition follows the Breitkopf \& Härtel edition in
      the first beat of measure 15.
    }{}

% TODO:
% This edition is Creative Commons
% licensed\footnote{\url{\doclicenseURL}} and the source code is
% available,\footnote{See
%   \url{https://github.com/dasch-swiss/Tondauer}.} to allow derived
% editions to be made.

\section*{About the Piece}

Mendelssohn composed this prelude in 1836, intending at first to
include it in Op.\ 35, a set of preludes (originally études) and
fugues. He then chose a different prelude for Op.\ 35, and this one
was published posthumously in 1868.\autocite[188--198]{todd_2008}

\ifdef{\critical}
  {% Text for the critical edition
    \section*{Methodology and Sources}

    Mendelssohn wrote two versions of the piece, \source{V}{1} and
    \source{V}{2}. Like most other editions, this one presents
    \source{V}{2}. No autograph (\source{A}{2}) or engraver's copy of
    \source{V}{2} has been found.\footnote{I am grateful to
      Dr. R. Larry Todd for this information.} I have used the
    following sources:
    
    \begin{description}
    \item[\source{A}{1}] The autograph of
      \source{V}{1},\footnote{\begin{otherlanguage}{ngerman}Staatsbibliothek
          zu Berlin\end{otherlanguage}, shelfmark
          Mus.ms.autogr. Mendelssohn Bartholdy, F. 28,
          \url{http://resolver.staatsbibliothek-berlin.de/SBB0001F9E700000315}.}
      dated 12 October 1836.
    \item[\source{E}{S1}] The first German edition of
      \source{V}{2},\footnote{\begin{otherlanguage}{ngerman}Staatsbibliothek
          zu Berlin\end{otherlanguage}, shelfmark N.Mus. 5420-1, and
          University of California Riverside Library, shelfmark SCUA
          M25.M45 P7.} published in Leipzig by Bartholf Senff in
      1868. It also gives the date of the composition as 12 October
      1836.
    \item[\source{E}{N}] The first English edition of
      \source{V}{2},\footnote{Bodleian Library, Oxford, shelfmark Deneke 256 (15).%
        % British Library, shelfmark Music Collections
        % R.M.26.h.6.(4.).
      } published in London by Novello
      in 1868.
    \item[\source{E}{S2}] A later Senff
      edition,\footnote{\begin{otherlanguage}{ngerman}Staatsbibliothek
          zu Berlin\end{otherlanguage}, shelfmark N.Mus. 5419-1.}
      published in about 1875. It is identical to \source{E}{S1}.
    \item[\source{E}{B}] The Breitkopf \& Härtel edition of
      \source{V}{2},\footnote{\url{https://imslp.org/wiki/Special:ReverseLookup/109142}}
      part of a critically revised edition of Mendelssohn's collected
      works, published between 1874 and 1877.
    \end{description}

    Digital facsimiles of these can be found at \url{https://tondauer.art}.

    \source{E}{S1} and \source{E}{N} were coordinated editions (each
    mentions the other publisher), but are not identical;
    \source{E}{N} contains several emendations. \source{E}{B} has some
    of these as well as a number of others.

    There are clearly problems with \source{E}{S1}. But it is possible
    that only the editors of \source{E}{S1} had access to the
    autographs, and that the variant readings in the other editions
    are conjectural emendations. Moreover, as explained below under
    \nameref{sec:analysis}, it seems likely that Mendelssohn left
    \source{A}{2} in an unfinished state, which may well be accurately
    reflected in \source{E}{S1}.

    I have taken \source{E}{S1} as a starting point, and accepted
    emendations from the other editions where they agree with
    \source{A}{1}. This approach gives considerable weight to
    \source{A}{1}, since it is the only available autograph, at the
    risk of undoing changes that Mendelssohn made in \source{A}{2}.
    I have also followed the emendation in \source{E}{N} and
    \source{E}{B} for what appears to be an error in measure 7, and
    supplied accidentals that are clearly missing. All these choices
    are detailed below under \nameref{sec:comments}.

    One could give more weight to \source{E}{S1}, given its proximity
    to \source{A}{2}. Or one could take a different approach and
    accept, for example, \source{E}{B}'s reading in the first beat of
    measure 15, on the grounds of musical plausibility.
    
    \section*{Comments}
    \label{sec:comments}

    Each comment corresponds to a footnote in the music text.

    \input{critical-print-score.annotations.inp}

    \section*{Analysis}
    \label{sec:analysis}

    In the process of transforming \source{V}{1} into \source{V}{2},
    Mendelssohn halved the note values throughout, and thus combined
    every two measures into one. Some of the problems with accidentals
    in \source{E}{S1} seem likely to have resulted from incomplete
    proofreading of the second half of each measure after this
    change.

    For example, in measure 8 of \source{A}{1}, the last note in the
    lower staff is A\na. There it did not need a natural sign, but it
    needed one after the preceding bar line was removed to form
    measure 4 of \source{V}{2}; the natural sign is missing in
    \source{E}{S1}. The same is true of the E\na\ notes in measure 26 of
    \source{A}{1}, which are in the third beat of measure 13 of
    \source{V}{2}, and the E\na\ in measure 30 of \source{A}{1},
    which is in the third beat of measure 15 in
    \source{V}{2}.

    Moreover, in measure 30 of \source{A}{1}, the first G\na\ in the
    lower staff has a cautionary accidental, but in \source{V}{2} it
    has been changed to another note. A natural sign therefore needs
    to be added to the subsequent G\na\ to cancel the sharp earlier in
    the measure, but it is missing in \source{E}{S1}.

    The redundant sharp on the E\sh\ in the third beat of measure 12
    in \source{E}{S1} seems to be another sign of this process; it was
    probably left there after being needed in measure 24 of
    \source{A}{1}. The same goes for the redundant sharp on the
    G\sh\ in the third beats of measures 16, 17, and 18, and the one
    on the E\sh\ in the third beat of measure 23.

    Perhaps Mendelssohn decided not to use this prelude in Op.\ 35
    before completing the proofreading of \source{A}{2}, and therefore
    left the manuscript in an unfinished state, with errors that were
    then reproduced in \source{E}{S1}.

    % The Novello edition in the British Library
    %
    % http://explore.bl.uk/BLVU1:LSCOP-ALL:BLL01004521054
    % https://www.bl.uk/digitisation-services
  }
  {}

\raggedbottom

\vspace{\baselineskip}

\hfill
\begin{minipage}[t]{0.55\textwidth}
  Benjamin Geer \\
  Data and Service Center for the Humanities \\
  University of Basel \\
  \href{mailto:benjamin.geer@dasch.swiss}{\smaller \texttt{benjamin.geer@dasch.swiss}} \\
  \texttt{\url{https://dasch.swiss}}
\end{minipage}

\pagebreak
\cleardoublepage
\mainmatter

% The PDF score rendered by LilyPond
\ifdef{\performance}
  {\ifdef{\variant}
    {\includepdf[pages=-]{performance-print-score-variant.pdf}}
    {\includepdf[pages=-]{performance-print-score.pdf}}}
  {\includepdf[pages=-]{critical-print-score.pdf}}
\end{document}
