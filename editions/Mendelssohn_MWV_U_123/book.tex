\documentclass[a4paper, 12pt]{book}

%%%%%%%%%%%%%%%%%%%%%%%%%%%%%%%%%%%%%%%%%%%%%%%%%%%%%%%%%%%%%%%%%%%%%%%%%%%%%%%%
% Packages

% Languages
\usepackage[french,ngerman,british]{babel}
\usepackage[autostyle,english=british]{csquotes}
\usepackage[super]{nth}

% Fonts
\usepackage{fontspec}
\usepackage{libertine}
\usepackage{anyfontsize}
\usepackage{musicography}

% Footnotes
\usepackage{scrextend}

% CC licence
\usepackage[
    type={CC},
    modifier={by-sa},
    version={4.0}
]{doclicense}

% Include PDF and EPS files
\usepackage[final]{pdfpages}
\usepackage{graphicx}

% Links
\usepackage[colorlinks=true, allcolors=blue]{hyperref}
\usepackage{relsize}
\usepackage{bookmark}

% Bibliography
\usepackage[backend=biber,notes,language=british]{biblatex-chicago}
\addbibresource{bibliography.bib}

%%%%%%%%%%%%%%%%%%%%%%%%%%%%%%%%%%%%%%%%%%%%%%%%%%%%%%%%%%%%%%%%%%%%%%%%%%%%%%%%
% Parsing of annotations extracted by scholarLY

\usepackage{pgfkeys}
\usepackage{ifthen}
\usepackage{etoolbox}
\usepackage{listofitems}
\setsepchar{,}

\pgfkeys{
  /ann/.is family,
  /ann,
  message/.store in=\annMessage,
  measure-no/.store in=\annMeasureNo,
  measure-pos/.store in=\annMeasurePos,
  beat-string/.store in=\annBeatString,
  beat-fraction/.store in=\annBeatFraction,
  beat-part/.store in=\annBeatPart,
  our-beat/.store in=\annOurBeat,
  rhythmic-location/.store in=\annRhythmicLocation,
  meter/.store in=\annMeter,
  source/.code={\pgfkeyssetvalue{source}{#1}\pgfkeysgetvalue{source}{\annSource}}
}

%%%%%%%%%%%%%%%%%%%%%%%%%%%%%%%%%%%%%%%%%%%%%%%%%%%%%%%%%%%%%%%%%%%%%%%%%%%%%%%%
% Formatting commands

% The name of a source
\newcommand{\source}[2]{\textbf{#1\textsubscript{#2}}}

% The \criticalRemark command used in annotations extracted by scholarLY
\newcommand{\criticalRemark}[1][]{%
  \pgfkeys{/ann,#1}% Parse the arguments
  \paragraph{\annMeasureNo, \annOurBeat}\pgfkeysifdefined{source}{\ifthenelse{\equal{\annSource}{}}{}{\readlist\annSourceList{\annSource}\annSource}}{}
  \par \annMessage
}

\newcommand{\bigdot}[0]{{\Large \textbullet}}

\newcommand{\centerbigdot}[0]{\begin{center}\bigdot\end{center}}

\begin{document}

\frontmatter

%%%%%%%%%%%%%%%%%%%%%%%%%%%%%%%%%%%%%%%%%%%%%%%%%%%%%%%%%%%%%%%%%%%%%%%%%%%%%%%%
% Title page

\begin{titlepage}
\begin{center}
  \fontspec{Quattrocento-Bold.ttf}{\fontsize{40}{50}\selectfont Felix Mendelssohn} \\
  \fontspec{Quattrocento-Regular.ttf}\vspace{2cm}
  {\fontsize{32}{42}\selectfont Prelude \textbullet\ \foreignlanguage{french}{Prélude}} \\
  \vspace{2 cm}
  {\fontsize{24}{34}\selectfont MWV U 123} \\
  \vspace{0.5 cm}
  {\fontsize{24}{34}\selectfont Op. 104a, No. 2} \\
  \vspace{4.6 cm}
  {\Large \ifdef{\critical}{Critical Edition \textbullet\ \foreignlanguage{french}{Édition critique}}{Performance Edition \textbullet\ \foreignlanguage{french}{Édition pratique}}} \\
  \vspace{0.5 cm}
  {\Large Edited by \textbullet\ \foreignlanguage{french}{préparée par} Benjamin Geer} \\
  \vspace{4.6 cm}
  {\footnotesize Version 0.10, 2021-04-22} \\
  \vspace{0.1 cm}
  {\footnotesize Copyright \copyright\ 2021 Benjamin Geer} \\
  \vspace{0.25 cm}
  \begin{minipage}{\textwidth}
  \centering
  $\vcenter{\hbox{\doclicenseImage[imagewidth=2cm]}}$
  \end{minipage}
  \end{center}
\end{titlepage}

%%%%%%%%%%%%%%%%%%%%%%%%%%%%%%%%%%%%%%%%%%%%%%%%%%%%%%%%%%%%%%%%%%%%%%%%%%%%%%%%
% Main document content

\chapter*{Preface \bigdot\ \foreignlanguage{french}{Préface}}

\section*{About the Piece \bigdot\ \foreignlanguage{french}{À propos de la pièce}}

Mendelssohn composed this prelude in 1836, intending at first to
include it in Op.\ 35, a set of preludes (originally études) and
fugues. He then chose a different prelude for Op.\ 35, and this one
was published posthumously in 1868.\footnote{\cite[188--198]{todd_2008}\label{todd}.}

\centerbigdot

\begin{otherlanguage}{french}
Mendelssohn a composé ce prélude en 1836, dans un premier temps pour
son op.\ 35, qui est un ensemble de préludes (qu'il a d'abord appelé
des études) et de fugues. Ensuite il a préféré un autre prélude pour
l'op.\ 35. Celui qui est présenté ici a été publié en 1868 après la
mort du compositeur.\footref{todd}
\end{otherlanguage}

\section*{About this Edition \bigdot\ \foreignlanguage{french}{À propos de cette édition}}

\ifdef{\critical}{% Text for the critical edition
    This critical edition is Creative Commons
    licensed\footnote{\url{\doclicenseURL}} and the source code is
    available,\footnote{\url{https://github.com/benjamingeer/Tondauer}\label{source-code}}
    to allow derived editions to be made.

    A performance edition based on this edition can be found on
    the project web site, \url{https://tondauer.art}.

    \centerbigdot

    \begin{otherlanguage}{french}
    Cette édition critique est diffusée sous licence Creative
    Commons\footnote{\url{https://creativecommons.org/licenses/by-sa/4.0/deed.fr}}
    et le code source est disponible,\footref{source-code} pour que
    des éditions dérivées puissent être réalisées.

    Une édition pratique basée sur cette édition se trouve
    sur le site Internet du projet, \url{https://tondauer.art}.
    \end{otherlanguage}
  }{% Text for the performance edition
    Footnotes in the music text point out editorial choices that are
    worth your attention, especially where you may prefer a different
    option. For explanations of these choices, please see the critical
    edition, which can be found on the project web site,
    \url{https://tondauer.art}.

    I would like to thank Penelope
    Roskell\footnote{\url{https://peneloperoskell.co.uk}\label{roskell}} for her
    advice on the suggested fingerings.

    This performance edition is Creative Commons
    licensed\footnote{\url{\doclicenseURL}} and the source code is
    available,\footnote{\url{https://github.com/benjamingeer/Tondauer}\label{source-code}}
    to allow derived editions to be made.

    \centerbigdot

    \begin{otherlanguage}{french}
    Les notes de base de page dans la partition signalent des choix
    éditoriaux qui méritent votre attention, surtout dans des cas où
    vous pourriez préférer une autre option. Vous trouverez des
    explications de ces choix dans l'édition critique, qui est
    disponible sur le site Internet du projet,
    \url{https://tondauer.art}.

    Je remercie Penelope Roskell\footref{roskell} pour ses conseils
    sur les doigtés proposés.

    Cette édition pratique est diffusée sous licence Creative
    Commons\footnote{\url{https://creativecommons.org/licenses/by-sa/4.0/deed.fr}}
    et le code source est disponible,\footref{source-code} pour que
    des éditions dérivées puissent être réalisées.
    \end{otherlanguage}
  }

\ifdef{\critical}
  {% Text for the critical edition
    \section*{Methodology \bigdot\ \foreignlanguage{french}{Méthodologie}}

    Mendelssohn wrote two versions of the piece, \source{V}{1} and
    \source{V}{2}. Like most other editions, this one presents
    \source{V}{2}. No autograph (\source{A}{2}) or engraver's copy of
    \source{V}{2} has been found.\footnote{I am grateful to
      Dr. R. Larry Todd for this information.} I have used the
    following sources:
    
    \begin{description}
    \item[\source{A}{1}] The autograph of
      \source{V}{1},\footnote{\begin{otherlanguage}{ngerman}Staatsbibliothek
          zu Berlin\end{otherlanguage}, shelfmark
          Mus.ms.autogr. Mendelssohn Bartholdy, F. 28,
          \url{http://resolver.staatsbibliothek-berlin.de/SBB0001F9E700000315}.}
      dated 12 October 1836.
    \item[\source{E}{S1}] The first German edition of
      \source{V}{2},\footnote{\begin{otherlanguage}{ngerman}Staatsbibliothek
          zu Berlin\end{otherlanguage}, shelfmark N.Mus. 5420-1.}
      published in Leipzig by Bartholf Senff in 1868. It also gives
      the date of the composition as 12 October 1836.
    \item[\source{E}{N}] The first English edition of
      \source{V}{2},\footnote{Bodleian Library, Oxford, shelfmark
        Deneke 256 (15).} published in London by Novello in 1868.
    \item[\source{E}{S2}] A later Senff
      edition,\footnote{\begin{otherlanguage}{ngerman}Staatsbibliothek
          zu Berlin\end{otherlanguage}, shelfmark N.Mus. 5419-1.}
      published in about 1875. It is identical to \source{E}{S1}.
    \item[\source{E}{B}] The Breitkopf \& Härtel edition of
      \source{V}{2},\footnote{\url{https://imslp.org/wiki/Special:ReverseLookup/109142}\label{imslp}}
      part of a critically revised edition of Mendelssohn's collected
      works, published between 1874 and 1877.
    \end{description}

    Digital facsimiles of these can be found at
    \url{https://tondauer.art}.

    \source{E}{S1} and \source{E}{N} were coordinated editions (each
    mentions the other publisher), but are not identical;
    \source{E}{N} contains several emendations. \source{E}{B} has some
    of these as well as a number of others.

    There are clearly problems with \source{E}{S1}. But it is possible
    that only the editors of \source{E}{S1} had access to the
    autographs, and that the variant readings in the other editions
    are conjectural emendations. Moreover, as explained below under
    \nameref{sec:analysis}, it seems likely that Mendelssohn left
    \source{A}{2} in an unfinished state, which may well be accurately
    reflected in \source{E}{S1}.

    I have taken \source{E}{S1} as a starting point, and accepted
    emendations from the other editions where they agree with
    \source{A}{1}. This approach gives considerable weight to
    \source{A}{1}, since it is the only available autograph, at the
    risk of undoing changes that Mendelssohn made in \source{A}{2}.  I
    have also followed the emendation in \source{E}{N} and
    \source{E}{B} for what apears to be an error in measure 7, and
    supplied accidentals that are clearly missing. All these choices
    are detailed below under \nameref{sec:comments}.

    One could give more weight to \source{E}{S1}, given its proximity
    to \source{A}{2}. Or one could take a different approach and
    accept, for example, \source{E}{B}'s reading in the first beat of
    measure 15, on the grounds of musical plausibility.

    \centerbigdot

    \begin{otherlanguage}{french}
    Mendelssohn a écrit deux versions de cette pièce~: \source{V}{1} et
    \source{V}{2}. Comme la plupart des autres éditions, celle-ci présente
    \source{V}{2}. Aucun autographe (\source{A}{2}) ou copie d'éditeur
    de \source{V}{2} a été trouvé.\footnote{Je suis reconnaissant au
      Dr. R. Larry Todd pour ce renseignement.} J'ai utilisé les sources
    suivantes :
    
    \begin{description}
    \item[\source{A}{1}] L'autographe de
      \source{V}{1},\footnote{\begin{otherlanguage}{ngerman}Staatsbibliothek
          zu Berlin\end{otherlanguage}, cote
          Mus.ms.autogr. Mendelssohn Bartholdy, F. 28,
          \url{http://resolver.staatsbibliothek-berlin.de/SBB0001F9E700000315}.}
      daté du 12 octobre 1836.
    \item[\source{E}{S1}] La première édition allemande de
      \source{V}{2},\footnote{\begin{otherlanguage}{ngerman}Staatsbibliothek
          zu Berlin\end{otherlanguage}, \foreignlanguage{french}{cote} N.Mus. 5420-1.}
      publiée à Leipzig par Bartholf Senff en 1868. La composition y est également
      datée du 12 octobre 1836.
    \item[\source{E}{N}] La première édition anglaise de
      \source{V}{2},\footnote{Bodleian Library, Oxford, \foreignlanguage{french}{cote}
        Deneke 256 (15).} publiée à Londres par Novello en 1868.
    \item[\source{E}{S2}] Une édition Senff
      ultérieure,\footnote{\begin{otherlanguage}{ngerman}Staatsbibliothek
      zu Berlin\end{otherlanguage}, \foreignlanguage{french}{cote} N.Mus. 5419-1.}
      publiée en 1875 environ. Elle est identique à \source{E}{S1}.
    \item[\source{E}{B}] L'édition Breitkopf \& Härtel de
      \source{V}{2},\footref{imslp}
      qui fait partie d'une édition revue et corrigée des \oe uvres de Mendelssohn,
      publiée entre 1874 et 1877.
    \end{description}

    Des fac-similés numériques de ces sources se trouvent sur le site
    \url{https://tondauer.art}.

    Les éditions \source{E}{S1} and \source{E}{N} ont été coordinées
    (chacune mentionne l'autre éditeur), mais ne sont pas identiques~:
    \source{E}{N} apporte plusieurs corrections. Plusieurs d'entre
    elles, ainsi que bien d'autres, se trouvent dans \source{E}{B}.

    Il est certain que \source{E}{S1} contient des erreurs. Mais il
    est possible que seuls les éditeurs de \source{E}{S1} aient eu
    accès aux autographes, et que les variantes présentes dans les
    autres éditions soient des conjectures. De plus, il est
    vraisemblable que Mendelssohn a laissé \source{A}{2} inachevé
    (voir \nameref{sec:analysis}) et que \source{E}{S1} ne fait que
    reproduire les erreurs de cet autographe.

    J'ai pris \source{E}{S1} comme point de départ. Là où les autres
    éditions présentent des leçons qui ne s'accordent pas avec
    \source{E}{S1}, j'ai préféré les leçons qui sont également
    présentes dans \source{A}{1}. Cela accorde un poids considérable
    à \source{A}{1}, parce que c'est le seul autographe disponible, au
    risque de ne pas tenir compte de modifications que Mendelssohn a
    pu apporter dans \source{A}{2}.  J'ai aussi accepté la correction,
    dans \source{E}{N} et \source{E}{B}, de ce qui semble être une
    erreur dans la mesure 7. J'ai également ajouté des accidents qui
    sont clairement nécessaires. Tous ces choix sont notés ci-dessous
    dans la section \nameref{sec:comments}.

    On pourrait accorder plus d'autorité à \source{E}{S1}, étant donné
    sa proximité à \source{A}{2}. Ou bien on pourrait suivre une autre
    logique et accepter, par exemple, la leçon de \source{E}{B} au
    premier temps de la mesure 15, au nom de sa plausibilité musicale.
    \end{otherlanguage}
    
    \section*{Comments \bigdot\ \foreignlanguage{french}{Remarques}}
    \label{sec:comments}

    Each comment corresponds to a footnote in the music text. The
    numbers preceding a comment identify the measure and beat that
    it refers to.  These are followed by the identifiers of the
    sources, if any, that contain the preferred reading.
    
    \centerbigdot

    \begin{otherlanguage}{french}
      Chaque remarque correspond à une note de bas de page dans la
      partition. Les numêros précédant une remarque identifient la
      mesure et le temps auxquels elle se réfère. Ceux-ci sont suivis,
      le cas échéant, par les identifiants des sources présentant la
      leçon préférée.
    \end{otherlanguage}

    \input{critical-score.annotations.inp}

    \section*{Analysis \bigdot\ \foreignlanguage{french}{Analyse}}
    \label{sec:analysis}

    In the process of transforming \source{V}{1} into \source{V}{2},
    Mendelssohn halved the note values throughout and removed half of
    the bar lines. Some of the problems with accidentals in
    \source{E}{S1} seem likely to have resulted from incomplete
    proofreading of the second half of each measure after this change.

    For example, in measure 8 of \source{A}{1}, the last note in the
    lower staff is \textit{A}\na. There it did not need a natural
    sign, but it needed one after the preceding bar line was removed
    to form measure 4 of \source{V}{2}; the natural sign is missing in
    \source{E}{S1}. The same is true of the \textit{E}\na\ notes in
    measure 26 of \source{A}{1}, which are in the third beat of
    measure 13 of \source{V}{2}, and the \textit{E}\na\ in measure 30
    of \source{A}{1}, which is in the third beat of measure 15 in
    \source{V}{2}.
    
    Moreover, in measure 30 of \source{A}{1}, the first
    \textit{G}\na\ in the lower staff has a cautionary accidental, but
    in \source{V}{2} (measure 15, third beat) it has been changed to
    another note. A natural sign therefore needs to be added to the
    subsequent \textit{G}\na\ to cancel the sharp earlier in the
    measure, but it is missing in \source{E}{S1}.

    The redundant sharp on the \textit{E}\sh\ in the third beat of
    measure 12 in \source{E}{S1} seems to be another sign of this
    process; it was probably left there after being needed in measure
    24 of \source{A}{1}. The same goes for the redundant sharp on the
    \textit{G}\sh\ in the third beats of measures 16, 17, and 18, and
    the one on the \textit{E}\sh\ in the third beat of measure 23.

    Perhaps Mendelssohn decided not to use this prelude in Op.\ 35
    before completing the proofreading of \source{A}{2}, and therefore
    left the manuscript in an unfinished state, with errors that were
    then reproduced in \source{E}{S1}.

    \centerbigdot

    \begin{otherlanguage}{french}
    En transformant \source{V}{1} en \source{V}{2}, Mendelssohn a
    divisé les durées par deux et supprimé la moitié des barres de
    mesure. Certains problèmes concernant les accidents dans
    \source{E}{S1} ont vraisemblablement résulté d'une relecture
    inachevée de la seconde moitié de chaque mesure après ce
    remaniement.

    Par exemple, dans la mesure 8 de \source{A}{1}, la dernière note
    de la portée inférieure est \textit{la}\na. Un bécarre, dont la
    note n'a pas besoin à cet endroit-là, est devenu nécessaire après
    que la barre de mesure précédente a été supprimée pour former la
    mesure 4 de \source{V}{2}, or ce bécarre manque dans
    \source{E}{S1}. Il en va de même pour les \textit{mi}\na\ de la
    mesure 26 de \source{A}{1}, qui se trouvent au troisième temps de
    la mesure 13 de \source{V}{2}, ainsi que pour le \textit{mi}\na\ de la
    mesure 30 de \source{A}{1}, qui se trouve au troisième temps de la
    mesure 15 dans \source{V}{2}.

    D'ailleurs, dans la mesure 30 de \source{A}{1}, il y a une
    altération de précaution devant le premier \textit{sol}\na\ de la
    portée inférieure, mais dans \source{V}{2} (measure 15, troisième
    temps) il y a une note différente à cet endroit-là. Un bécarre
    doit donc être ajouté devant le \textit{sol}\na\ suivant pour
    annuler le dièse précédent, mais ce bécarre manque dans
    \source{E}{S1}.

    Le dièse superflu devant le \textit{mi}\sh\ du troisième temps de
    la mesure 12 dans \source{E}{S1} semble être encore un effet de ce
    processus.  Il y est probablement resté parce qu'il était
    necessaire dans la mesure 24 de \source{A}{1}. Il en va de même
    pour le dièse superflu devant le \textit{sol}\sh\ du troisième
    temps des mesures 16, 17 et 18, ainsi que celui devant le
    \textit{mi}\sh\ du troisième temps de la mesure 23.
    
    Il se peut que Mendelssohn ait decidé de ne pas inclure ce prélude
    dans l'op.\ 35 avant de terminer la relecture d'\source{A}{2}, et
    qu'il ait donc laissé le manuscrit inachevé, sans avoir corrigé
    des erreurs que \source{E}{S1} a reproduit par la suite.
    \end{otherlanguage}
  } {}

\raggedbottom

\vspace{\baselineskip}

\hfill
\begin{minipage}[t]{0.55\textwidth}
  Benjamin Geer \\
  \url{https://benjamingeer.name}
\end{minipage}

\pagebreak
\cleardoublepage
\mainmatter

% The PDF score rendered by LilyPond
\ifdef{\performance}
      {\includepdf[pages=-]{performance-score.pdf}}
      {\includepdf[pages=-]{critical-score.pdf}}
\end{document}
