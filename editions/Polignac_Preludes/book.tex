\documentclass[a4paper, 12pt]{book}

%%%%%%%%%%%%%%%%%%%%%%%%%%%%%%%%%%%%%%%%%%%%%%%%%%%%%%%%%%%%%%%%%%%%%%%%%%%%%%%%
% Packages

% Languages
\usepackage[ngerman,british,french]{babel}
\usepackage[autostyle,english=british]{csquotes}
\usepackage[super]{nth}

% Fonts
\usepackage{microtype}
\usepackage{fontspec}
\usepackage{libertine}
\usepackage{anyfontsize}
\usepackage{musicography}

% Footnotes
\usepackage{scrextend}

% CC licence
\usepackage[
    type={CC},
    modifier={by-sa},
    version={4.0}
]{doclicense}

% Include PDF and EPS files
\usepackage[final]{pdfpages}
\usepackage{graphicx}

% Links
\usepackage[colorlinks=true, allcolors=blue]{hyperref}
\usepackage{relsize}
\usepackage{bookmark}

% Bibliography
\usepackage{hanging}

%%%%%%%%%%%%%%%%%%%%%%%%%%%%%%%%%%%%%%%%%%%%%%%%%%%%%%%%%%%%%%%%%%%%%%%%%%%%%%%%
% Parsing of annotations extracted by scholarLY

\usepackage{pgfkeys}
\usepackage{ifthen}
\usepackage{etoolbox}
\usepackage{listofitems}
\setsepchar{,}

\pgfkeys{
  /ann/.is family,
  /ann,
  message/.store in=\annMessage,
  measure-no/.store in=\annMeasureNo,
  measure-pos/.store in=\annMeasurePos,
  beat-string/.store in=\annBeatString,
  beat-fraction/.store in=\annBeatFraction,
  beat-part/.store in=\annBeatPart,
  our-beat/.store in=\annOurBeat,
  rhythmic-location/.store in=\annRhythmicLocation,
  meter/.store in=\annMeter,
  source/.code={\pgfkeyssetvalue{source}{#1}\pgfkeysgetvalue{source}{\annSource}}
}

%%%%%%%%%%%%%%%%%%%%%%%%%%%%%%%%%%%%%%%%%%%%%%%%%%%%%%%%%%%%%%%%%%%%%%%%%%%%%%%%
% Formatting commands

% The name of a source
\newcommand{\source}[2]{\textbf{#1\textsubscript{#2}}}

% The \criticalRemark command used in annotations extracted by scholarLY
\newcommand{\criticalRemark}[1][]{%
  \pgfkeys{/ann,#1}% Parse the arguments
  \paragraph{\annMeasureNo, \annOurBeat}\pgfkeysifdefined{source}{\ifthenelse{\equal{\annSource}{}}{}{\readlist\annSourceList{\annSource}\annSource}}{}
  \par \annMessage
}

\newcommand{\bigdot}[0]{{\Large \textbullet}}

\newcommand{\centerbigdot}[0]{\begin{center}\bigdot\end{center}}

\begin{document}

\frontmatter

%%%%%%%%%%%%%%%%%%%%%%%%%%%%%%%%%%%%%%%%%%%%%%%%%%%%%%%%%%%%%%%%%%%%%%%%%%%%%%%%
% Title page

\begin{titlepage}
\begin{center}
  \fontspec{Quattrocento-Bold.ttf}{\fontsize{36}{46}\selectfont Armande de Polignac} \\
  \fontspec{Quattrocento-Regular.ttf}\vspace{3.25 cm}
  {\fontsize{30}{40}\selectfont Préludes pour piano} \\
  \vspace{0.5 cm}
  {\fontsize{30}{40}\selectfont \textbullet} \\
  \vspace{0.35 cm}
  {\fontsize{30}{40}\selectfont Preludes for Piano} \\
  \vspace{3.25 cm}
  {\Large Édition critique \textbullet\ Critical Edition} \\
  \vspace{3.25 cm}
  {\Large Préparée par \textbullet\ \foreignlanguage{english}{edited by}} \\
  \vspace{0.25 cm}
  {\Large Benjamin Geer \& Florence Launay} \\
  \vspace{3.25 cm}
  {\footnotesize Version 1.0, 2022-10-29} \\
  \vspace{0.25 cm}
  \begin{minipage}{\textwidth}
  \centering
  $\vcenter{\hbox{\doclicenseImage[imagewidth=2cm]}}$
  \end{minipage}
  \end{center}
\end{titlepage}

%%%%%%%%%%%%%%%%%%%%%%%%%%%%%%%%%%%%%%%%%%%%%%%%%%%%%%%%%%%%%%%%%%%%%%%%%%%%%%%%
% Main document content

\chapter*{Préface \bigdot\ \foreignlanguage{english}{Preface}}

Les \emph{Préludes} font partie des premières œuvres publiées par
Armande de Polignac, vers 1900. Aux côtés de ses premières mélodies,
éditées en 1898, de son premier quatuor à cordes, joué en concert en
1899, et de son \emph{Ouverture de Lear} pour orchestre, créée en 1902 à
Montreux, ils marquent les débuts d'une production musicale qui comprend
plus de cent soixante-dix œuvres, dans tous les genres musicaux alors
pratiqués, avec une préférence marquée pour les ballets. Sa production
la place parmi les compositrices importantes du début du
XX\textsuperscript{e}~siècle, cette époque qui voit une floraison
inédite de la composition par les femmes dans le monde entier, sous
l'influence des mouvements d'émancipation qui ont fait évoluer les
mentalités déniant aux femmes des capacités créatrices et qui ont
notamment favorisé leur accès à des études de composition dans les
conservatoires.

Armande de Polignac naît à Paris le 8 janvier 1876 dans une famille
importante de l'aristocratie française. Elle ne connaîtra pas sa mère,
Marie Langenberger, d'origine allemande, décédée une semaine après sa
naissance, dont on sait qu'elle était une violoniste amatrice de haut
niveau. Son père, Camille de Polignac, se remarie en 1883 avec une jeune
Anglaise, Margaret Knight de Wolwerley, et c'est à Londres que l'enfant
reçoit sa première formation musicale. Un de ses propres curriculums vitae
évoque ces années~: «~Armande de Polignac montra dès l'enfance une
passion pour la musique. À deux ans, elle s'asseyait aux pieds de son
oncle, le prince Edmond de Polignac (qui épousa Winnaretta Singer) pour
l'écouter improviser au piano. À quatre ans, elle cherchait des mélodies
sur le piano~; vers neuf ou dix ans, elle avait pour Bach une telle
passion qu'elle cherchait à convertir les domestiques à cette adoration.
À onze ans, étudiant depuis longtemps le piano et le violon, elle
commença les études d'harmonie à Londres avec des maîtres allemands~;
elle composait depuis l'enfance. À dix-sept ans, elle eut la révélation
de la musique russe et vint travailler à Paris, d'abord avec Gigout
(organiste à Saint-Augustin) pour le contrepoint et la fugue, puis avec
Fauré, enfin à la Schola de d'Indy, où elle tenait en outre dans
l'orchestre la partie d'alto, et où elle travailla la composition,
l'orchestration, la direction d'orchestre~». Ce travail de la direction
avec Vincent d'Indy lui permet de diriger certaines de ses œuvres, comme
son opéra \emph{La Petite Sirène} en 1907 à l'Opéra de Nice et son
ballet \emph{Les Mille et une Nuits} en 1914 au Théâtre du Châtelet, et
de rejoindre ainsi la poignée de femmes qui ont pu accéder à la
direction d'orchestre à cette époque, notamment d'autres compositrices
comme Nadia Boulanger, Cécile Chaminade, Juliette Folville, Louise
Héritte-Viardot et Augusta Holmès. De ses propres dires, Armande de
Polignac avait «~voué sa vie à la musique~», précisant~: «~Je ne reçois
jamais, je ne vais jamais dans le monde, à moins d'y être appelée
professionnellement. Car je ne suis pas une mondaine qui compose de la
musique à ses moments perdus et pour se distraire. Je suis une femme qui
a appris un métier, après avoir fait l'apprentissage nécessaire~». Elle
épouse en 1895 le comte Alfred de Chabannes La Palice~; ils ont une
fille unique, Hedwige. Le mariage ne freine pas ses activités. Elle peut
compter sur le soutien de son époux, très mélomane, qui partage son
désir de se faire connaître sous son nom de jeune fille. Armande de
Polignac réussit à s'imposer comme compositrice professionnelle malgré
les préjugés, doubles dans son cas~: à la fois femme et aristocrate,
elle se retrouve classée parmi les dilettantes.

Sa production pour le piano comprendrait une cinquantaine de pièces dont
une dizaine seulement ont été publiées. L'imprécision découle du fait
qu'un catalogue des œuvres
d'Armande de Polignac n'a pas encore été réalisé. Certaines de ses
pièces sont apparentées au genre de la musique légère, comme les
\emph{Danses mièvres} (1902) et la \emph{Danse persane} (1922). D'autres
présentent une écriture de piano ambitieuse qui témoigne de son niveau
pianistique élevé (elle interprétait parfois ses propres pièces en
concert), comme ses \emph{Échappées} qui ont été publiées en 1909 et
qu'elle a dédiées au célèbre pianiste Ricardo Viñes, ses \emph{Prancing
Goddesses} et ses \emph{Féeries nocturnes} (deux œuvres non datées et
restées manuscrites). Les \emph{Préludes} révèlent quant à eux un
parti-pris d'anti-virtuosité, tout en témoignant d'une recherche très
personnelle bien éloignée de l'esprit de la «~musique de salon~» si
présente dans la vie musicale de cette époque. Armande de Polignac est à
cette époque sous l'influence de ses études à la Schola Cantorum, où
l'emphase est mise sur la redécouverte du patrimoine musical des siècles
précédents~: à l'écriture dépouillée, presque ascétique, empreints de
nostalgie, ils se présentent comme des harmonisations hardies de thèmes
rappelant parfois les claviéristes français du passé. On remarque
l'éventuelle influence d'Erik Satie dans l'accompagnement syncopé à la
main gauche qui parcourt le \emph{Prélude} II~; et de Gabriel Fauré dans
l'exposition des thèmes des \emph{Préludes} IV et VI, où l'harmonie est
révélée par des arpèges à la main gauche toujours précédés d'un
demi-soupir. Les \emph{Préludes} I et II ont été utilisés par la
compositrice comme soutien à des poèmes, sous forme de
mélodrame/adaptation, un genre populaire au tournant des
XX\textsuperscript{e}~et XX\textsuperscript{e}~siècles~: son exemplaire
de la partition porte les titres respectifs de \emph{Printemps mort} et
\emph{Berceuse} ainsi que quelques répliques, clairement de sa main.

Le pianiste Laurent Martin, qui est la première personne à avoir rejoué
les \emph{Préludes} au XXI\textsuperscript{e}~siècle, a témoigné dans le
livret accompagnant son enregistrement paru en 2019~: «~Elle a assimilé
la musique du passé au cours de ses études, mais, dans ses créations,
elle va exprimer sa forte personnalité, d'abord plus discrète dans ses
\emph{Six préludes} du début du XX\textsuperscript{e}~siècle qui sont
des miniatures concentrées et disent tout en une ou deux minutes,
élégance, fantaisie, sensibilité, nostalgie, originalité~».

\section*{Bibliographie}

\subsection*{En français}

\begin{hangparas}{.5cm}{1}

  Florence Launay, \emph{Les Compositrices en France au
  XIX\textsuperscript{e}~siècle}, Paris, Fayard, 2006.
  
  Florence Launay, «~Armande de Polignac~», \emph{Compositrices
  françaises au XX\textsuperscript{e}~siècle}, Paris, Delatour, 2007,
  p.~177-189.

  Trotier, Rémy-Michel, «~La tendre et tumultueuse musique
  d'Armande de Polignac pour les ballets de Loïe Fuller~», \emph{Euterpe},
  n°~38, mai 2022, p.~4-13.
  
\end{hangparas}

\vspace{\baselineskip}

\subsection*{En anglais}

\begin{hangparas}{.5cm}{1}
  
  Laura Hamer, «~Armande de Polignac: An Aristocratic
  \emph{Compositrice} in \emph{Fin-de-siècle} Paris~», dans
  \emph{Women in the arts in the Belle Epoque: Essays on influential
  artists, writers and performers}, Jefferson, McFarland \& Company,
  2012, p.~165-185.
  
\end{hangparas}

\vspace{\baselineskip}

\section*{Discographie}

\begin{hangparas}{.5cm}{1}
  
  \emph{Préludes}, dans \emph{Compositrices d'exception}, Laurent
  Martin, piano, Ligia Digital, 2019.

  \emph{Préludes pour piano}, dans \emph{Armande de Polignac : Mélodies, Préludes pour piano},
  Stéphanie Humeau, piano, Maguelone, 2022.

\end{hangparas}

\vspace{\baselineskip}

\centerbigdot

\begin{otherlanguage}{english}

The \emph{Preludes} date from about 1900, and are among the first
works that Armande de Polignac published. Along with her first songs
(published in 1898), her first string quartet (performed in 1899), and
her \emph{Ouverture de Lear [Overture to Lear]} for orchestra (first
performed in 1902 in Montreux), they represent the early period in a
body of work that includes more than a hundred and seventy pieces in all
the musical genres that were then in use, with a marked preference for
ballets. This output places her among the more important female
composers of the early \nth{20} century, a time that witnessed the
appearance of an unprecedented number of female composers the world
over, under the influence of women's liberation movements, which were
successfully challenging attitudes that denied women's creativity. In
particular, women's movements had made it easier for women to enrol in
composition courses at conservatories.

Armande de Polignac was born on 8 January 1876 in a prominent French
aristocratic family. She did not know her mother, Marie Langenberger,
who was of German orgin, is known to have been a skilled amateur
violinist, and died a week after her birth. In 1883, Armande's father,
Camille de Polignac, married a young Englishwoman, Margaret Knight of
Wolwerley, and Armande received her first musical training in
London. According to one of her autobiographical essays, `Armande de
Polignac's enthusiasm for music was evident in early childhood. At the
age of two, she sat at the feet of her uncle, Prince Edmond de
Polignac (who married Winaretta Singer) to listen to him improvise at
the piano. When she was four, she picked out melodies at the piano. At
nine or ten, she had such a passion for Bach that she tried to instil
it in the servants. At eleven, after years of piano and violin
lessons, she began to study harmony with German teachers in London. By
this time she already composed music. At seventeen, she took an avid
interest in Russian music, and went to study in Paris, first with
Gigout (organist at the Saint Augustin Church) for counterpoint and
fugue, then with Fauré, and finally at d'Indy's Schola Cantorum, where
she played viola in the orchestra and studied composition,
orchestration, and conducting.' The conducting experience she gained
with Vincent d'Indy enabled her to conduct some of her own works, such
as her opera \emph{La Petite Sirène [The Little Mermaid]} in 1907 at
the Nice Opera, and her ballet \emph{Les Mille et une Nuits [The
  Thousand and One Nights]} in 1914 at the Théâtre du Châtelet. Thus
she joined the handful of women who had the opportunity to conduct an
orchestra during that period, most of whom were other female
composers, such as Nadia Boulanger, Cécile Chaminade, Juliette
Folville, Louise Héritte-Viardot, and Augusta Holmès. Armande de
Polignac had, as she put it, `devoted her life to music', adding: `I
never have guests at home, and I don't go to social gatherings, unless
it's in a professional capacity. I'm not a socialite who composes
music in her free time, to entertain herself. I'm a woman who has
learned a trade, having received the necessary training.' In 1895 she
married Count Alfred de Chabannes La Palice. They had one child,
Hedwige. Marriage did not slow down her musical activity. She could
rely on the support of her husband, an ardent music lover, who
approved of her wish to gain recognition under her maiden
name. Armande de Polignac succeeded in making her mark as a
professional composer despite the double prejudice she faced: as a
woman and an aristocrat, she was labelled a dilettante.

She seems to have written some fifty pieces for the piano, only about
ten of which have been published.
These numbers are imprecise because no catalogue of her works
has yet been produced. Some of her compositions can be considered light
music, such as \emph{Danses mièvres [Sentimental Dances]} (1902) and
\emph{Danse persane [Persian Dance]} (1922). Others display ambitious
piano writing, testifying to her high level of musicianship (she
sometimes performed her own pieces in concerts), such as
\emph{Échappées [Escapes]}, published in 1909 and dedicated to the
famous pianist Ricardo Viñes, \emph{Prancing Goddesses}, and
\emph{Féeries nocturnes [Nocturnal enchantments]} (the last two of
which are undated and exist only in manuscript form).

As for the \emph{Preludes}, they reflect a desire to avoid virtuosity,
while following an artistic path far removed from the spirit of the
`salon music' of the time. Armande de Polignac was then under the
influence of her training at the Schola Cantorum, which emphasised the
rediscovery of the musical heritage of previous centuries. The style
of the \emph{Preludes} is uncluttered, almost ascetic, and tinged with
nostalgia; they strike the listener as daring harmonisations of themes
that sometimes recall the French \emph{claviéristes} of the past. The
possible influence of Erik Satie can be detected in the left-hand
accompaniment of \emph{Prelude} II, and that of Gabriel Fauré in the
exposition of the themes in \emph{Preludes} IV and VI, in which the
harmony is revealed by arpeggios in the left hand, always preceded by
a quaver rest. The composer used \emph{Preludes} I and II to accompany
poems, in the form of melodramas/adaptations, a popular genre at the
turn of the \nth{19} and \nth{20} centuries: on her own copy of the
sheet music, the titles \emph{Printemps mort [Dead Spring]} and
\emph{Berceuse [Lullaby]}, respectively, along with a few lines of
text, are written in what is clearly her handwriting.

The pianist Laurent Martin, who was the first to play the
\emph{Preludes} in the \nth{21}~century, observed in the liner notes
of his recording, which appeared in 2019: `She absorbed the music of
the past in the course of her studies, but in her creations, she
expressed her own strong personality, at first in a subdued manner in
her \emph{Six Preludes} at the beginning of the \nth{20} century; they
are concentrated miniatures that say a great deal in one or two
minutes, with elegance, imagination, sensitivity, nostalgia, and
originality'.

\section*{Bibliography}

\subsection*{In English}

\begin{hangparas}{.5cm}{1}

  Hamer, Laura. `Armande de Polignac: An Aristocratic
  \emph{Compositrice} in \emph{Fin-de-siècle} Paris'. In \emph{Women
  in the arts in the Belle Epoque: Essays on influential artists,
  writers and performers}, edited by Paul Freyer, 165-185. Jefferson:
  McFarland \& Company, 2012.

\end{hangparas}

\vspace{\baselineskip}

\subsection*{In French}

\begin{hangparas}{.5cm}{1}

  Launay, Florence. \emph{Les Compositrices en France au
  XIX\textsuperscript{e}~siècle}. Paris: Fayard, 2006.
  
  \pagebreak
  
  Launay, Florence. `Armande de Polignac'. In \emph{Compositrices
  françaises au XX\textsuperscript{e}~siècle}, 177-189. Paris:
  Delatour, 2007.

  Trotier, Rémy-Michel. `La tendre et tumultueuse musique d'Armande de Polignac
  pour les ballets de Loïe Fuller'. \emph{Euterpe}, no. 38 (May 2022): 4-13.
    
\end{hangparas}

\vspace{\baselineskip}

\section*{Discography}

\begin{hangparas}{.5cm}{1}
  
  \emph{Préludes}. In \emph{Compositrices d'exception}. Laurent
  Martin, piano. Ligia Digital, 2019.

  \emph{Préludes pour piano}. In \emph{Armande de Polignac: Mélodies, Préludes pour piano}.
  Stéphanie Humeau, piano. Maguelone, 2022.

\end{hangparas}

\vspace{\baselineskip}

\hfill
\begin{minipage}[t]{0.55\textwidth}
  Florence Launay \\
  {\small \url{http://www.compositrices19.net}}
\end{minipage}

\end{otherlanguage}

\chapter*{À propos de cette édition
  \bigdot\ \foreignlanguage{english}{About this Edition}}

Cette édition fait partie du projet
Tondauer.\footnote{\url{https://tondauer.art}\label{project}} Elle est
diffusée sous licence Creative Commons\footnote{\url{\doclicenseURL}}
et le code source est
disponible.\footnote{\url{https://github.com/benjamingeer/Tondauer}\label{source-code}}

Nous sommes reconnaissants à feu Mme la comtesse Desvernay et à sa famille,
qui ont donné leur autorisation à la réalisation cette édition et qui nous ont permis
de consulter leurs archives.

\centerbigdot

\begin{otherlanguage}{english}
  This edition is part of the Tondauer project.\footref{project} It is
  Creative Commons
  licensed\footnote{\url{https://creativecommons.org/licenses/by-sa/4.0/deed.en}}
  and the source code is available.\footref{source-code}

  We are grateful to the late Countess Desvernay and to her family, who gave us
  permission to produce this edition and allowed us to consult their archives.
\end{otherlanguage}
    
\section*{Sources}
  Il ne semble pas subsister de manuscrit des \emph{Préludes}. Une édition du
  premier \emph{Prélude}, publiée par E.\ Baudoux et Cie.\ à Paris en 1901,
  est conservée à la Bibliothèque nationale de France
  (\source{E}{B}).\footnote{\url{https://catalogue.bnf.fr/ark:/12148/cb43209797j}\label{BnF}}
  L'édition des six \emph{Préludes} publiée par Bellon, Ponscarme et Cie. à
  Paris entre 1903 et 1906 (\source{E}{P}) ne semble se trouver dans
  aucune archive publique, mais les descendants d'Armande de Polignac
  en possèdent un exemplaire. Des fac-similés numériques de ces
  sources sont disponibles sur le site Internet du
  projet.\footref{project}
  
  Nous avons corrigé des erreurs apparentes (voir
  \nameref{sec:comments}) et suivi la pratique courante en ce qui
  concerne les altérations de précaution.
  
  \centerbigdot
  
  \begin{otherlanguage}{english}
    No manuscript of the \emph{Preludes} seems to have survived. An edition
    of the first \emph{Prelude}, published by E.\ Baudoux et Cie.\ in 1901,
    is preserved at the Bibliothèque nationale de France
    (\source{E}{B}).\footref{BnF} The edition of the six \emph{Preludes}
    published by Bellon, Ponscarme et Cie.\ in Paris between 1903 and
    1906 (\source{E}{P}) does not seem to be in any public archive,
    but the descendants of Armande de Polignac possess a copy of
    it. Digital facsimiles of these sources can be found on the
    project web site.\footref{project}

    We have corrected apparent errors (see \nameref{sec:comments}), and
    followed common practice in the use of cautionary accidentals.
  \end{otherlanguage}
  
  \section*{Remarques \bigdot\ \foreignlanguage{english}{Comments}}
  \label{sec:comments}

    Chaque remarque correspond à une note de bas de page dans la
    partition. Les numéros précédant une remarque identifient la
    mesure et le temps auxquels elle se réfère. Ceux-ci sont suivis,
    le cas échéant, par l'identifiant de la source dans laquelle se
    trouve la version préférée. \source{E}{P} est la seule source pour
    tous les \emph{Préludes} sauf le premier.

    \centerbigdot

    \begin{otherlanguage}{english}
      Each comment corresponds to a footnote in the music text. The
      numbers preceding a comment identify the measure and beat that
      it refers to. These are followed by the identifier of the
      source, if any, that contains the preferred
      reading. \source{E}{P} is the only source for all the \emph{Preludes}
      except the first.
    \end{otherlanguage}
    
    \subsection*{Prélude I}
    
    \input{prelude-1-critical-score.annotations.inp}

    \subsection*{Prélude II}

    \input{prelude-2-critical-score.annotations.inp}
    
    \subsection*{Prélude III}

    \input{prelude-3-critical-score.annotations.inp}

    %\subsection*{Prélude IV}

    %\input{prelude-4-critical-score.annotations.inp}

    \subsection*{Prélude V}

    \input{prelude-5-critical-score.annotations.inp}

    \subsection*{Prélude VI}

    \input{prelude-6-critical-score.annotations.inp}

\raggedbottom

\vspace{\baselineskip}

\hfill
\begin{minipage}[t]{0.55\textwidth}
  Benjamin Geer \\
  \url{https://benjamingeer.name}
\end{minipage}

\pagebreak
\cleardoublepage
\mainmatter

% The PDF score rendered by LilyPond
\includepdf[pages=-]{prelude-1-critical-score.pdf}
        \includepdf[pages=-]{prelude-2-critical-score.pdf}
        \includepdf[pages=-]{prelude-3-critical-score.pdf}
        \includepdf[pages=-]{prelude-4-critical-score.pdf}
        \includepdf[pages=-]{prelude-5-critical-score.pdf}
        \includepdf[pages=-]{prelude-6-critical-score.pdf}
\end{document}
